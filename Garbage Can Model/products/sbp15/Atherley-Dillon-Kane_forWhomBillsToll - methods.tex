%This research paper---as opposed to a “narrative/composition” paper (e.g., for an English course)---is a written report based on a systematic analysis that you conduct on a topic concerning Complexity Theory in the Social Sciences. Your analysis will apply a selection of concepts, theories, models, or other ideas covered in this course. The actual analysis comes first; the paper comes second, so think of this paper as a “lab report:” you write it up after you have conducted your experiment. Do not begin writing this paper before you have finished (or almost finished) your analysis, otherwise you will likely encounter serious problems. As a general guideline, papers like this are usually 15-20 pages in length. Double-space the entire paper, except the Bibliography.
%
%Use any major citation style (Turabian, MLA, Chicago); just be consistent throughout.
%
%Use any major citation style (Turabian, MLA, Chicago); just be consistent throughout.
\documentclass[pdftex,12pt]{llncs}
\usepackage[american]{babel}
\usepackage{wrapfig}
% \newcommand{\x}[1]{ }
% This is where the bibliography stuff needs to happen
\usepackage[style=apa,backend=biber]{biblatex}
\DeclareLanguageMapping{english}{american-apa}
%\addbibresource{billTolls.bib}
%\addbibresource{Projects-garbageCanCongress.bib}
\usepackage[utf8]{inputenc}
\usepackage{csquotes} % context sensitive quotes ---makes this look good
\usepackage[pdftex]{graphicx}
\usepackage{xfrac}
\usepackage{cleveref}

\begin{document}

\section{Methods}
%(3-4 pages) Write this section third!
%
%This section identifies and defines those concepts, models, theories, or other course ideas and tools that you used in order to carry out your analysis. Which concepts, data, theories, principles or other analytical tools from the course—from lectures or readings—did you use in this study? Which sources?
%
%Which data processing procedures? Which cases or samples? Which time periods (epochs)? Key:reproducibility! I must be able to replicate your findings, based on the information provided in this section.

\subsection{Model Overview}
In this section, we describe implementation of a model of policy formation through simulated annealing.
Conceptually, a bill is sponsored by a single legislator, reviewed and revised through two rounds of simulated annealing (by a network of the sponsor's friends and a committee of interested members), and finally voted on by the entire legislative body.
Legislators are initialized with priorities and positions on a common list of issues, with preferences stochastically generated according to quantified state priorities and party ideologies.
Legislators are connected to each other in a social network, generated using both homophily \parencite{msc01, br11} and preferential attachment \parencite{Barabasi1999}.
The initialization and simulation processes are described further in sections below.
\footnote{In the interest of brevity, we have omitted some detail descriptions of the initialization and simulation; however, these details may be found in the online supplement at TBD and are referenced where applicable below.}

\subsection{Model Initialization}
Each model realization is initialized by generating default issue priorities and positions via the \texttt{State} object, legislators with heterogeneous priorities and positions, and a social network among the legislators.

\subsubsection{Initializing the Model Environment}
For our model, we assume that several core issues represent powerful, crystallizing factors that differentiate our simulated parties.
Thus, party "platforms" consist of vectors of positions and priorities on a set of issues that includes both \texttt{State\_Priorities} and a random sample of high-priority \texttt{Ideology\_Issues} (TBD: S?).
These vectors are used as "seeds" for the stochastic generation of individual legislator preferences.

\subsubsection{Legislator Initialization}
Each legislator's issue priorities are assigned with a stochastic preferential attachment method  (TBD: S?) to the seed values provided by the \texttt{State} object, generating a power law distribution of priorities for that legislator, and providing some correlation in legislator priorities to the extent that seed vectors are correlated through state priorities and party ideology.
We assume that party-affiliated legislators adopt the positions of their party; for all other issues (and all issues for unaffiliated legislators), positions are assigned uniform randomly from the range of allowed position values ($2^4$ for our model).

The end-result of this process is a set of legislator agents with heterogeneous but correlated policy preferences as conditioned by party ideologies and state priorities, and with the strength of correlation determined by party-alignment.

\subsubsection{Network Generation}
Model initialization is completed with a social network designed to capture the social dynamics of legislating, in that a Congressmen will naturally have a set of friends and close colleagues that he interacts with more than others. 
The network is generated using preferential attachment and total homophily over the generated preferences.
 
Preferential attachment is as described in Barabasi and Albert (1999), with $m=5$ new edges selected randomly from a \textit{pdf} of degree distribution in the sub-network of potential friends of each legislator. 
The set of potential friends is selected using an issue-priority weighted homophily over all issues (TBD: S?).

The typical outcome of this procedure is a network among legislator agents with "small-world" properties (TBD cite Watts and Strogatz), consistent with existing research on social networks in Congress \parencite{Granovetter1978}.

\subsection{Simulation Algorithm Overview}
Having defined a population of legislators and their relationship to each-other, we next establish a procedure for legislators to engage in the business of law-making.\footnote{One might argue that this is a departure from realism, as the current Congress does not appear to do this.
However, we are attempting to generalize a model of law-making in legislatures.
Some legislatures do legislate, periodically.}
In our model of law-making, the simulation sequentially repeats the following process for 200 proposals (or halts if all issues are passed into law):

\begin{enumerate}
\item Proposal:
\begin{enumerate}
\item A random legislator is chosen to sponsor a bill.
\item The sponsor proposes a draft bill on any issue that has not already been addressed by law.  This initial draft reflects the sponsor's position on that issue.
\end{enumerate}
\item Draft circulation among cosponsors:
\begin{enumerate}
\item All legislators connected to the sponsor in the social network are selected as cosponsors.
\item The cosponsors revise the draft using simulated annealing; new issues may be added to the bill during the revision process and solutions on existing issues may change (TBD: S?).
\end{enumerate}
\item Committee review:
\begin{enumerate}
\item The draft is referred to a committee, reflecting committee agenda-setting powers \parencite{cm93, cm05}.
Legislators for whom the main issue of the bill is a high priority are assigned to the relevant committee (TBD: S?).
\item The committee revises the bill by simulated annealing; again, new issues may be added and existing solutions changed.
\end{enumerate}
\item Floor vote:
\begin{enumerate}
\item The bill is referred to the floor for a vote.
\item A legislator votes `yes' to a bill when her satisfaction with it is greater than the model parameter \texttt{satisfaction\_threshold}.
\item If the bill passes by simple majority (\textgreater 50\% votes), the bill is made into law; \textit{i.e.}, the solutions addressed by the bill are recorded and the issues may not be revisited for the remainder of the realization.
\end{enumerate}
\end{enumerate}

\subsubsection{Simulated Annealing}
Bill revision is implemented by the Metropolis algorithm for simulated annealing \parencite{mrrt53, kgv}.
Our energy function is the cumulative dissatisfaction of all reviewers over all dimensions of the bill (TBD S?).  Increases of $0.1$ in dissatisfaction were accepted with probability \sfrac{1}{2} at the maximum temperature (higher satisfaction energy states are automatically accepted).
The annealing schedule is discussed in TBD: S?.

\subsection{Model Calibration}
We calibrated legislators' \textit{satisfaction thresholds}---the point at which they vote "yes" on legislation---to achieve a ~4\% pass-rate, comparable to passage rates in the actual US Congress (between 2\% and 7\% in recent history).\footnote{Pass rates are equal to the total number of bills passed in a given Congress divided by the total counts of introduced legislation for that Congress. Data to calculate pass rates was collected from Civic Impulse LLC (http://www.govtrack.us).\label{passfn}}

\subsection{Experiments}
Table TBD identifies the model parameters and values over which a suite of experiments was run.
A factorial design was used to select combinations of values on which to run experiments.
30 realizations were run for each combination of values.
To keep the data set manageable, the run history for only a single realization was recorded, as a sample, for each experiment.
However, the following metrics were recorded for each realization:  number of laws passed, number of issues addressed by law, and legislative body satisfactions over all bills before and after SA revisions.
Aggregate statistics (averages and standard deviations) were also calculated and recorded for the output metrics of all realizations of an experiment.

(TBD Table Here)

\end{document}